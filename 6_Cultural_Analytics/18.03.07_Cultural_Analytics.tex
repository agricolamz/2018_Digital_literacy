\documentclass[13pt, t]{beamer}

% Presento style file
\usepackage{config/presento}
\usepackage{ulem}

% custom command and packages
\input{config/custom-command}

% Information
\title{\huge Cultural Analytics}
\author[shortname]{Г. Мороз }
\date{\begin{center} 
\large 7 марта 2018 г.
\end{center}}

\begin{document}

\begin{frame}[plain]
\maketitle
\end{frame}

\framecard[colorblue]{{\color{colorwhite} \huge Большие данные\\ в исследованиях культуры}}

\begin{frame}{Project Gutenberg}
\begin{itemize}
\item \href{/}{\alert {https://www.gutenberg.org/}}
\item основана Майклом Хартом в 1971 году
\item доступ к 56 000 книг, в самом разном формате
\item поиск по категориям и мета-данным
\item API
\end{itemize}
\vfill
\begin{center}
\includegraphics[width=0.5\linewidth]{images/01-gutenberg.jpg}
\end{center}
\end{frame}

\begin{frame}{Europeana Collections}
\begin{itemize}
\item \href{https://www.europeana.eu}{\alert {https://www.europeana.eu}}
\item доступ к 50 000 000 предметов материальной культуры
\begin{itemize}
\item  более 27 000 000 изображений
\item  более 22 000 000 текстов
\item  более 1 000 000 видео
\item  более 700 000 звукозаписей
\item  ...
\end{itemize}
\item поиск по категориям и мета-данным
\item API
\end{itemize}
\vfill
\begin{center}
\includegraphics[width=0.3\linewidth]{images/02-europeana.png}
\end{center}
\end{frame}

\begin{frame}{Digital Public Library of America}
\begin{itemize}
\item \href{https://dp.la/}{\alert {https://dp.la/}}
\item поиск по категориям и мета-данным
\item представление не только в виде списка, но и в виде временной ленты и карты
\item много приложений
\item API
\end{itemize}
\vfill
\begin{center}
\includegraphics[width=0.5\linewidth]{images/03-dpla.png}
\end{center}
\end{frame}

\begin{frame}{HathiTrust digital library}
\begin{itemize}
\item \href{https://www.hathitrust.org/}{\alert {https://www.hathitrust.org/}}
\item поиск по категориям и мета-данным
\item охват большего количества стран
\item API
\end{itemize}
\vfill
\begin{center}
\includegraphics[width=0.25\linewidth]{images/04-hathiTrust.png}
\end{center}
\pause
Hathi [hatiː] переводится с хинди `слон', животное, котрое славится своей памятью.
\end{frame}

\begin{frame}{Российская государственная библиотека}
\begin{itemize}
\item \href{https://www.rsl.ru/}{\alert {https://www.rsl.ru/}}
\item  \sout{поиск по категориям и мета-данным}
\item  \sout{API}
\end{itemize}
\vfill
\begin{center}
\includegraphics[width=0.25\linewidth]{images/05-rsl.jpeg}
\end{center}
\end{frame}

\begin{frame}{Диджитализация музеев}
\begin{itemize}
\item экспонаты переезжают в интернет, например, \href{https://www.museum-digital.de}{https://www.museum-digital.de}
\item музей переезжает в интернет, например, \href{http://www.louvre.fr/en/visites-en-ligne\#tabs}{Лувр}, \href{http://www.britishmuseum.org/with_google.aspx}{Британский музей}, \href{http://www.nga.gov/exhibitions/webtours.htm}{Национальная галерея искусства в Вашингтоне}
\item экспозиции дополняются при помощи технологий
\begin{itemize}
\item аудиогид
\item QR-коды
\item очки дополненной реальности
\end{itemize}
\end{itemize}
\end{frame}

\framepic{images/06-google-ngrams.png}{
{\large Культуромика (colturomics) ---} еще один пример, как сделать новой наукой хорошо забытое старое}

\framecard[colorblue]{{\color{colorwhite} \huge Как все это исследовать?}}

\begin{frame}{Cultural Analysis Lab}
\begin{itemize}
\item \href{http://manovich.net/}{\alert {http://manovich.net/}}
\item  основным идиологом исследований в области культурной аналитики стал Лев Манович
\end{itemize}
\end{frame}

\framecard[colorblue]{{\color{colorwhite} \huge Спасибо за внимание! \bigskip\\
\Large Пишите письма\\
agricolamz@gmail.com}}

\end{document}