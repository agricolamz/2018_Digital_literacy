\documentclass[13pt, t]{beamer}

% Presento style file
\usepackage{config/presento}
\usepackage{ulem}

% custom command and packages
\input{config/custom-command}

% Information
\title{\huge 
14. Компьютерное моделирование\\
15. Работа с аудио- и видеоматериалами 
}
\author[shortname]{Г. Мороз }
\date{\begin{center} 
\large 1 июня 2018 г.
\end{center}}

\begin{document}

\begin{frame}[plain]
\maketitle
\end{frame}

\framecard[colorblue]{{\color{colorwhite} \huge Компьютерное моделирование}}

\begin{frame}{Компьютерное моделирование}
построение модели, описывающее поведение какой-то системы
\begin{itemize}
\item статистическая модель 
\hfill → обсуждали на лекции про маш. обуч.
\item симуляция
\item \textit{модель}, описывающая форму некоторого объекта, здания и т. п.
\end{itemize}
\vfill
\pause Но если подумать все это концептуально почти одно и то же.\\
Мы будем обсуждать 2D и 3D моделирование.
\end{frame}

\begin{frame}{2D и 3D модели}
\begin{itemize}
\item \pause ... живопись и скульптура \pause
\item компьютерные изображения
\begin{itemize}
\item растровые
\item векторные
\end{itemize}
\end{itemize}
\end{frame}

\begin{frame}{Растровые vs. Векторные}
\begin{multicols}{2}
\begin{itemize}
\item плохо масштабируются
\item больше весят
\item всемогущие
\end{itemize}
\columnbreak
\begin{itemize}
\item легко масштабируются
\item меньше весят
\item имеют ограничения
\end{itemize}
\end{multicols}
Можно конвертировать друг в друга.
\end{frame}

\framepic{images/01-Raster-vs-Vector.png}

\framepic{images/02-vectorised-photo}

\begin{frame}{Анимация}
\begin{itemize}
\item от руки
\item  анимация
\item  фотоперекладка
\item  стоп-моушен
\item  компьютерная анимация
\end{itemize}
\end{frame}

\framepic{images/03-illusionist}{\Large \href{https://www.youtube.com/watch?v=d5TglK8v1xs}{Сильвен Шоме (2010) Иллюзионист}}

\framepic{images/04-a-scanner.png}{\Large \href{https://www.youtube.com/watch?v=TY5PpGQ2OWY}{Ричард Линклейтер (2006) Помутнение}}

\framepic{images/05-isle-of-dogs.jpg}{\Large \href{https://www.youtube.com/watch?v=dt__kig8PVU}{Уэс Андерсон (2018) Остров собак}}

\framepic{images/06-up.jpg}{\Large \href{https://www.youtube.com/watch?v=pkqzFUhGPJg}{Пит Доктер (2009) Вверх}}

\begin{frame}{Computer-assisted vs. computer-generated}
Существует несколько стратегий работы в компьютерной графике:
\begin{itemize}
\item computer-assisted
\item computer-generated
\end{itemize}
\end{frame}

\framepic{images/07-road-to-el-dorado.jpeg}{\Large \href{https://www.youtube.com/watch?v=JcOfJwN0bdY}{Бибо Бергерон (2000) Дорога в Эльдорадо}}

\framepic{images/08-toy-story.jpg}{\Large \href{https://www.youtube.com/watch?v=tN1A2mVnrOM}{Джон Лассетер (1995) История игрушек}}

\framepic{images/09-3d-printing.jpg}{\medskip \color{colorwhite}  \Large Теперь вы видите, что это --- лишь технология\vfill \pause ... начинку уже придумали}

\framecard[colorblue]{{\color{colorwhite} \huge Работа с аудио- и видеоматериалами}}

\begin{frame}{Какие задачи стоят при анализе аудио- и видео-}
\pause
\begin{itemize}
\item извлечение, соединение, редактирование фрагментов
\item извлечение некоторой информации
\begin{itemize}
\item спектральная информация
\begin{itemize}
\item тишина vs. разговоры/музыка
\item один инструмент vs. оркестр
\item светлый кадр vs. темный
\end{itemize}
\item семантическая информация
\begin{itemize}
\item помещение vs. улица
\item распознавание чего-нибудь (человек, лицо, дым, движение и т. п.)
\item время дня
\item распознавание речи
\end{itemize}
\item оценка положения
\begin{itemize}
\item egomotion --- оценка положения двигающейся камеры на основании ``увиденного''/``услышанного'' (парктроник, дроны)
\end{itemize}
\end{itemize}
\end{itemize}
\end{frame}

\begin{frame}{Какие методы?}
\begin{itemize}
\item ручная разметка
\item → обученные алгоритмы, опирающиеся на спектр
\item → нейросети
\end{itemize}

Иногда при анализе помогает предобработка данных:
\begin{itemize}
\item обрезка каких-то частот
\item перевод видео в ЧБ
\end{itemize}
\end{frame}

\framecard[colorblue]{{\color{colorwhite} \huge Спасибо за внимание! \bigskip\\
\Large Пишите письма\\
agricolamz@gmail.com}}

\end{document}